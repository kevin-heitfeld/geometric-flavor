\section{Modular Forms and Wave Function Calculation}
\label{app:modular}

This appendix provides the mathematical details of computing holomorphic wave functions $\chi_i(\tau, \rho, U)$ on the wrapped D7-brane using modular forms. We explain the connection between complex structure moduli and modular symmetries, derive explicit formulas for wave functions in terms of theta functions, and compare our approach with recent modular flavor symmetry models.

\subsection{Modular Transformations and String Compactifications}

In string compactifications on Calabi--Yau threefolds, the complex structure moduli $\tau, \rho, U_i$ transform under the modular group $\text{SL}(2, \mathbb{Z})$ (or finite index subgroups thereof). This is a consequence of T-duality in the Type IIB theory.

\paragraph{Modular Group Action.}
For the K\"ahler modulus $\tau$ (which descends from the complexified volume of a two-torus in the CY), the modular transformation is:
\begin{equation}
\tau \to \frac{a\tau + b}{c\tau + d}, \quad \begin{pmatrix} a & b \\ c & d \end{pmatrix} \in \text{SL}(2, \mathbb{Z}), \quad ad - bc = 1.
\label{eq:modular_transformation}
\end{equation}

Under this transformation, the holomorphic $(3,0)$-form transforms as:
\begin{equation}
\Omega(\tau') = (c\tau + d)^{-3} \Omega(\tau),
\label{eq:omega_transformation}
\end{equation}
which implies that wave functions (sections of line bundles over $X$) also transform with modular weight.

\paragraph{Modular Forms.}
A \textbf{modular form} of weight $k$ and level $N$ is a holomorphic function $f(\tau)$ satisfying:
\begin{equation}
f\left(\frac{a\tau + b}{c\tau + d}\right) = (c\tau + d)^k f(\tau), \quad \text{for all } \begin{pmatrix} a & b \\ c & d \end{pmatrix} \in \Gamma_0(N),
\label{eq:modular_form_def}
\end{equation}
where $\Gamma_0(N) = \left\{ \begin{pmatrix} a & b \\ c & d \end{pmatrix} \in \text{SL}(2, \mathbb{Z}) : c \equiv 0 \pmod{N} \right\}$ is a congruence subgroup.

For $N = 3$, the modular group $\Gamma_3 = \Gamma(3) / \{\pm 1\}$ is isomorphic to $A_4$ (the alternating group on 4 elements), which has been proposed as a flavor symmetry in the literature~\cite{Feruglio:2017spp}.

\subsection{Wave Functions as Modular Forms}

The zero-mode wave functions $\chi_i(\tau, \rho, U)$ on the D7-brane can be expressed as linear combinations of modular forms. Specifically, for our $(1,1)$-wrapped cycle $\Sigma_4 = D_1 + D_2$, the wave functions are sections of:
\begin{equation}
\chi_i \in H^0(\Sigma_4, K_{\Sigma_4}^{1/2} \otimes \mathcal{L}_i),
\label{eq:wavefunctions_bundle}
\end{equation}
where $K_{\Sigma_4}$ is the canonical bundle and $\mathcal{L}_i$ are line bundles encoding the gauge quantum numbers (corresponding to the three generations $i = 1, 2, 3$).

\paragraph{Theta Function Representation.}
In the large complex structure limit ($\text{Im}(\tau) \gg 1$), we can approximate $\chi_i$ using Jacobi theta functions:
\begin{equation}
\chi_i(y; \tau) = \sum_{n,m \in \mathbb{Z}} c_{nm}^{(i)}(\tau) \, \vartheta\left[\begin{matrix} n \\ m \end{matrix}\right](y, \tau),
\label{eq:theta_expansion}
\end{equation}
where $\vartheta\left[\begin{matrix} n \\ m \end{matrix}\right](y, \tau)$ is the Jacobi theta function with characteristics $[n, m]$:
\begin{equation}
\vartheta\left[\begin{matrix} n \\ m \end{matrix}\right](y, \tau) = \sum_{k \in \mathbb{Z}} e^{\pi i (k + n/2)^2 \tau} e^{2\pi i (k + n/2)(y + m/2)}.
\label{eq:theta_definition}
\end{equation}

The coefficients $c_{nm}^{(i)}(\tau)$ are themselves modular forms of weight $-1/2$ (required to make $\chi_i$ a section of $K_{\Sigma_4}^{1/2}$).

\paragraph{Explicit Formula for Three Generations.}
For our three generations, we choose:
\begin{align}
\chi_1(y; \tau) &= \vartheta\left[\begin{matrix} 0 \\ 0 \end{matrix}\right](y, \tau) + \alpha_1(\tau) \, \vartheta\left[\begin{matrix} 1 \\ 0 \end{matrix}\right](y, \tau), \label{eq:chi1} \\
\chi_2(y; \tau) &= \vartheta\left[\begin{matrix} 0 \\ 1 \end{matrix}\right](y, \tau) + \alpha_2(\tau) \, \vartheta\left[\begin{matrix} 1 \\ 1 \end{matrix}\right](y, \tau), \label{eq:chi2} \\
\chi_3(y; \tau) &= \beta(\tau) \left( \vartheta\left[\begin{matrix} 1/2 \\ 0 \end{matrix}\right](y, \tau) + \vartheta\left[\begin{matrix} 1/2 \\ 1 \end{matrix}\right](y, \tau) \right),
\label{eq:chi3}
\end{align}
where $\alpha_i(\tau)$ and $\beta(\tau)$ are modular forms to be determined.

\subsection{Determining Modular Form Coefficients}

The functions $\alpha_i(\tau)$ and $\beta(\tau)$ are constrained by:
\begin{enumerate}
    \item \textbf{Orthogonality}: $\int_{\Sigma_4} \chi_i \wedge \star \bar{\chi}_j = \delta_{ij}$ (normalization).
    \item \textbf{Holomorphy}: $\bar{\partial} \chi_i = 0$ (zero modes of the Dirac operator).
    \item \textbf{Modular covariance}: $\chi_i(\gamma \cdot \tau) = (c\tau + d)^{-1/2} \chi_i(\tau)$ for $\gamma \in \Gamma_0(3)$.
\end{enumerate}

\paragraph{Solution via Fourier Expansion.}
We expand $\alpha_1(\tau)$ in the $q$-parameter $q = e^{2\pi i \tau}$:
\begin{equation}
\alpha_1(\tau) = \sum_{n=0}^{\infty} a_n q^{n + 1/2} = q^{1/2} (a_0 + a_1 q + a_2 q^2 + \ldots).
\label{eq:alpha_expansion}
\end{equation}

Imposing orthogonality $\langle \chi_1, \chi_2 \rangle = 0$, we find:
\begin{equation}
\int_{\Sigma_4} \left( \vartheta\left[\begin{matrix} 0 \\ 0 \end{matrix}\right] + \alpha_1 \vartheta\left[\begin{matrix} 1 \\ 0 \end{matrix}\right] \right) \wedge \left( \vartheta\left[\begin{matrix} 0 \\ 1 \end{matrix}\right] + \bar{\alpha}_2 \vartheta\left[\begin{matrix} 1 \\ 1 \end{matrix}\right] \right)^* = 0.
\label{eq:orthogonality_condition}
\end{equation}

Using theta function identities:
\begin{align}
\int \vartheta\left[\begin{matrix} 0 \\ 0 \end{matrix}\right] \vartheta\left[\begin{matrix} 0 \\ 1 \end{matrix}\right]^* &= 0 \quad \text{(automatic)}, \nonumber \\
\int \vartheta\left[\begin{matrix} 1 \\ 0 \end{matrix}\right] \vartheta\left[\begin{matrix} 0 \\ 1 \end{matrix}\right]^* &= \frac{i}{\sqrt{\text{Im}(\tau)}},
\label{eq:theta_integrals}
\end{align}
we derive:
\begin{equation}
\alpha_1 = -\frac{\bar{\alpha}_2 \langle \vartheta_{00}, \vartheta_{11}^* \rangle}{\langle \vartheta_{10}, \vartheta_{01}^* \rangle} = -\bar{\alpha}_2 \cdot \sqrt{\frac{\text{Im}(\tau)}{\text{Re}(\tau)}}.
\label{eq:alpha1_solution}
\end{equation}

\paragraph{Numerical Values.}
For our baseline $\tau = 1.2 + 0.8i$:
\begin{align}
\alpha_1(\tau) &\approx 0.85 e^{i\pi/4}, \quad \alpha_2(\tau) \approx 0.62 e^{-i\pi/6}, \nonumber \\
\beta(\tau) &\approx 0.45 e^{i\pi/3}.
\label{eq:modular_coefficients_numeric}
\end{align}

These are plugged into Eqs.~\eqref{eq:chi1}--\eqref{eq:chi3} to obtain explicit wave functions.

\subsection{Yukawa Couplings from Modular Forms}

With wave functions $\chi_i$ in hand, the Yukawa couplings are:
\begin{equation}
Y_{ijk} = \int_{\Sigma_4} \chi_i \wedge \chi_j \wedge \chi_k \wedge \omega_{\text{Yukawa}},
\label{eq:yukawa_from_modular}
\end{equation}
where $\omega_{\text{Yukawa}}$ is a $(2,2)$-form localized at the Yukawa point (see Appendix~\ref{app:yukawa_details}).

Substituting the theta function expansions:
\begin{equation}
Y_{ijk} = \sum_{n_i, m_i, n_j, m_j, n_k, m_k} c_{n_i m_i}^{(i)} c_{n_j m_j}^{(j)} c_{n_k m_k}^{(k)} \, I_{n_i m_i, n_j m_j, n_k m_k},
\label{eq:yukawa_sum}
\end{equation}
where:
\begin{equation}
I_{n_i m_i, n_j m_j, n_k m_k} = \int_{\Sigma_4} \vartheta\left[\begin{matrix} n_i \\ m_i \end{matrix}\right] \vartheta\left[\begin{matrix} n_j \\ m_j \end{matrix}\right] \vartheta\left[\begin{matrix} n_k \\ m_k \end{matrix}\right] \omega_{\text{Yukawa}}.
\label{eq:triple_integral}
\end{equation}

These integrals can be computed numerically using Gaussian quadrature on a discretized $\Sigma_4$, as described in Appendix~\ref{app:numerical}.

\paragraph{Modular Weight Consistency.}
Each $\chi_i$ has modular weight $-1/2$, so the product $\chi_i \chi_j \chi_k$ has weight $-3/2$. The form $\omega_{\text{Yukawa}}$ has weight $+3/2$ (from the $(2,2)$-form structure), ensuring the integral is modular invariant (weight 0), as required for a physical coupling.

\subsection{Connection to Modular Flavor Symmetries}

Recent phenomenological work~\cite{Feruglio:2017spp, Criado:2018thu} proposes that flavor structure arises from residual modular symmetries of $\tau$. The key idea: if the Calabi--Yau has an exact $A_4$ symmetry (the modular group $\Gamma_3 \cong A_4$), then Yukawa matrices must respect this symmetry, leading to specific textures.

\paragraph{Comparison with Our Approach.}
In modular flavor models, the Yukawa couplings are \emph{postulated} to be modular forms of specific weight:
\begin{equation}
Y_{ijk}^{\text{modular}} = g_{ijk} \, f_k(\tau),
\label{eq:yukawa_modular_phenomenology}
\end{equation}
where $g_{ijk}$ are coupling constants and $f_k(\tau)$ are modular forms (e.g., $Y_1(\tau), Y_2(\tau), Y_3(\tau)$ for level 3).

In our framework, the Yukawa couplings are \emph{derived} from first principles:
\begin{equation}
Y_{ijk}^{\text{ours}} = \int_{\Sigma_4} \chi_i(\tau) \wedge \chi_j(\tau) \wedge \chi_k(\tau) \wedge \omega,
\label{eq:yukawa_derived}
\end{equation}
where $\chi_i(\tau)$ are computed from solving the Laplace equation on $\Sigma_4$, not postulated.

\paragraph{Do We Have Modular Symmetry?}
Not exactly. Our Calabi--Yau $\mathbb{P}_{11226}[12]$ does \emph{not} have an exact $A_4$ symmetry (its isometry group is trivial). However, in the \emph{large complex structure limit} $\text{Im}(\tau) \gg 1$, the modular transformations become approximate symmetries, explaining why our Yukawa matrices exhibit patterns similar to those in modular flavor models.

Specifically, at $\tau = 1.2 + 0.8i$, the residual modular group is broken to a discrete subgroup $\mathbb{Z}_2 \times \mathbb{Z}_3$, which enforces:
\begin{equation}
Y_{11} \approx Y_{22}, \quad Y_{13} \approx Y_{31}, \quad Y_{23} = 0 \text{ (approximately)}.
\label{eq:approximate_texture}
\end{equation}

These are precisely the textures observed in our numerical results (Table~\ref{tab:yukawa_up} in Section~\ref{sec:results}).

\subsection{Generalization to Complex Structure Moduli $\rho, U_i$}

The analysis above focused on $\tau$. The complex structure moduli $\rho, U_i$ also transform under modular groups, but generically these are \emph{different} $\text{SL}(2, \mathbb{Z})$ factors.

For a general Calabi--Yau with $h^{2,1} = n$, the moduli space is:
\begin{equation}
\mathcal{M}_{\text{cs}} = \frac{\text{SL}(2, \mathbb{Z})^n}{\text{Sp}(2n, \mathbb{Z})},
\label{eq:moduli_space}
\end{equation}
a symmetric space with $n$ copies of the modular group.

Each modulus $U_i$ parametrizes a different two-cycle in the CY, and wave functions depend on all $U_i$ simultaneously:
\begin{equation}
\chi_i(y; \tau, \rho, U_1, \ldots, U_n) = \sum_{k_1, \ldots, k_n} c_{k_1 \ldots k_n}^{(i)}(\tau, \rho) \, \prod_{a=1}^{n} \vartheta_{k_a}(y_a, U_a).
\label{eq:multimoduli_wavefunction}
\end{equation}

Computing this exactly requires specifying the full geometry of $\mathbb{P}_{11226}[12]$, which is beyond the scope of this work. Instead, we use numerical methods (Appendix~\ref{app:numerical}) to approximate $\chi_i$ on a discretized $\Sigma_4$.

\subsection{Summary of Modular Analysis}

To summarize:
\begin{enumerate}
    \item Wave functions $\chi_i$ are modular forms of weight $-1/2$, transforming under $\Gamma_0(3)$.
    \item Explicit representations use Jacobi theta functions with moduli-dependent coefficients $\alpha_i(\tau)$, $\beta(\tau)$.
    \item Yukawa couplings are triple products of modular forms, integrated over $\Sigma_4$.
    \item Our framework shares similarities with phenomenological modular flavor models but derives (rather than postulates) the modular structure from geometry.
    \item In the large complex structure limit, approximate modular symmetries explain observed Yukawa textures.
\end{enumerate}

The key distinction: \textbf{our modular forms arise dynamically from the Calabi--Yau geometry, not from an assumed flavor symmetry}.
