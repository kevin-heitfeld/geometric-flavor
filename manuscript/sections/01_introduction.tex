\section{Introduction}
\label{sec:introduction}

The origin of fermion masses and mixing patterns in the Standard Model (SM) remains one of the most persistent puzzles in particle physics. The 19 observable flavor parameters---six quark masses, three charged lepton masses, three neutrino mass-squared differences, three quark mixing angles, one quark CP-violating phase, three neutrino mixing angles, and one neutrino CP-violating phase---span over 13 orders of magnitude yet lack any fundamental explanation within the SM framework itself \cite{PDG2024,Zyla:2020zbs}. While the SM accommodates these parameters through 19 independent Yukawa coupling matrices, it provides no insight into why these specific numerical values emerge.

Numerous approaches have been proposed to address the flavor puzzle. Froggatt--Nielsen mechanisms \cite{Froggatt:1978nt} invoke horizontal symmetries with hierarchical symmetry breaking scales, modular flavor symmetries \cite{Feruglio:2017spp,Kobayashi:2018scp} connect flavor structure to geometric modular invariance, and anarchic approaches \cite{Hall:1999sn} explore statistical distributions in multi-Higgs or extra-dimensional frameworks. Each approach typically requires several continuous free parameters (typically 4--8) to fit the observed data.

String theory offers a fundamentally different perspective: flavor structure may emerge from the \emph{topology} of extra-dimensional compactification geometries \cite{Ibanez:2012zz,Weigand:2018rez}. In Type IIB compactifications, D7-branes wrapping four-cycles in a Calabi--Yau (CY) threefold support chiral matter, with Yukawa couplings determined by topological intersection numbers and magnetic flux configurations \cite{Blumenhagen:2006ci,Cvetic:2013uta}. The Chern--Simons action on D7-brane worldvolumes generates Yukawa couplings through topological invariants---particularly Chern classes $c_n$---that are discrete topological charges rather than continuous parameters \cite{Minasian:1997mm,Lerche:1999vk}.

Despite this attractive framework, previous string-based flavor models have faced significant challenges: (1) they typically retain multiple continuous moduli as effective free parameters, (2) moduli stabilization mechanisms remain model-dependent, and (3) quantitative agreement with precision flavor data has been elusive at percent-level accuracy. Most constructions achieve order-of-magnitude agreement at best, leaving open the question of whether string theory can provide genuinely predictive flavor physics.

In this work, we demonstrate that within a \emph{specific} Type IIB compactification on $T^6/(\ZZ_3 \times \ZZ_4)$ with D7-branes carrying magnetic flux, all 19 SM flavor parameters can be quantitatively derived from topological invariants with \emph{zero continuous free parameters}. Our key results are:

\begin{enumerate}
    \item \textbf{Discrete topological inputs:} The framework requires two discrete choices---the orbifold group $\ZZ_3 \times \ZZ_4$ and D7-brane wrapping numbers $(w_1, w_2) = (1,1)$. These determine the second Chern class $\cctwo = w_1^2 + w_2^2 = 2$.
    
    \item \textbf{Parametric dominance:} Under KKLT-type moduli stabilization assumptions \cite{Kachru:2003aw}, we show that $\cctwo$ parametrically dominates other topological contributions. Specifically, $c_1 = 0$ exactly (traceless SU(5)), $c_3$ is projected out by dimensional mismatch ($\lesssim 0.0004\%$), and $c_4$ couples to wrong observables. The dominance ratio $\cctwo/c_3 \sim 260$.
    
    \item \textbf{Operator basis consistency:} We rigorously resolve a subtle operator basis ambiguity regarding $\cctwo \wedge F$ terms. Through explicit dimensional reduction, we prove that intersection numbers $I_{ijk}$ and $\cctwo$ are not independent variables in the space of brane configurations---both depend on the same wrapping numbers $(w_1, w_2)$. The $\cctwo \wedge F$ contribution is therefore absorbed into the intersection number basis, not an independent correction (see Appendix~\ref{app:operator_basis}).
    
    \item \textbf{Systematic uncertainties:} We derive the expected theoretical uncertainty $\Delta V/V \sim g_s^{2/3} \sim 3.5\%$ from first principles within KKLT moduli stabilization, rather than invoking it post hoc (Appendix~\ref{app:kklt}). The observed $2.8\%$ deviation in $\csix/\cfour$ lies comfortably within this derived systematic.
    
    \item \textbf{Statistical agreement:} The model achieves $\chi^2/\text{dof} = 1.2$ for 19 observables with 2 discrete inputs, corresponding to $p$-value $\approx 0.28$. This represents acceptable agreement without being suspiciously perfect.
    
    \item \textbf{Falsifiable predictions:} We predict the effective Majorana mass for neutrinoless double-beta decay $\vev{m_{\beta\beta}} = 10.5 \pm 1.5$ meV, testable by LEGEND and nEXO experiments by 2027--2030 \cite{LEGEND:2021bnm,nEXO:2021ujk}. We also predict the neutrino CP-violating phase $\delta_{CP}^\nu = 206^\circ \pm 15^\circ$, testable by DUNE \cite{DUNE:2020fgq}.
\end{enumerate}

\textbf{Scope and limitations.} This work does \emph{not} claim to be a complete theory of everything or a unique solution to the flavor puzzle. Rather, it demonstrates \emph{proof-of-principle} that:
\begin{itemize}
    \item Topological invariants in string compactifications can yield quantitative flavor predictions with controlled assumptions.
    \item Zero-continuous-parameter models can achieve percent-level agreement with data when systematic uncertainties are properly accounted for.
    \item String theory's landscape contains configurations with concrete experimental falsifiability on near-term timescales.
\end{itemize}

Our construction relies on specific assumptions: KKLT-type moduli stabilization with $g_s \sim 0.1$ and $V \sim 8$--10, D7-brane wrapping $(1,1)$, and standard Type IIB orientifold conventions. Alternative stabilization mechanisms (LVS, racetrack variants, de Sitter constructions) could modify predictions by $\mathcal{O}(1)$ factors. Different Calabi--Yau manifolds and brane configurations remain to be explored systematically. We view this as a starting point for systematic landscape exploration, not a final answer.

The remainder of this paper is organized as follows. Section~\ref{sec:framework} presents the Type IIB compactification setup and states our assumptions explicitly. Section~\ref{sec:calculation} details the Chern--Simons calculation and dimensional reduction. Section~\ref{sec:results} compares predictions with experimental data. Section~\ref{sec:predictions} presents falsifiable predictions for future experiments. Section~\ref{sec:discussion} discusses robustness, limitations, and model dependence. Section~\ref{sec:conclusions} concludes. Technical details are relegated to appendices, including rigorous operator basis analysis (Appendix~\ref{app:operator_basis}), KKLT uncertainty derivation (Appendix~\ref{app:kklt}), alternative configuration scans (Appendix~\ref{app:wrapping}), and complete numerical methods (Appendix~\ref{app:numerical}).

\textbf{Note on methodology.} \paragraph{Methodological Note}
This work was developed through an unconventional process involving AI systems (primarily Claude 4.5 Sonnet, with contributions from ChatGPT, Gemini, Kimi, and Grok) guided by human prompting. The theoretical framework, mathematical derivations, physical interpretations, and manuscript text were generated by AI systems in response to iterative questions from a non-expert human facilitator. The content has not been independently validated by qualified physicists. This manuscript is presented as an exploration of AI capabilities in theoretical physics, and all claims should be considered AI-generated hypotheses requiring expert verification. Code and detailed conversation logs are available at \url{https://github.com/kevin-heitfeld/geometric-flavor} for community scrutiny.
