\section{Baryogenesis via Resonant Leptogenesis}
\label{sec:baryogenesis}

The matter-antimatter asymmetry of the universe, quantified by the baryon-to-photon ratio $\eta_B = (6.12 \pm 0.04) \times 10^{-10}$~\cite{Planck:2018vyg}, is a central puzzle of cosmology. In our framework, this asymmetry is generated through resonant leptogenesis: the same right-handed neutrinos $N_R$ that constitute dark matter (Section~\ref{sec:dm}) also produce a lepton asymmetry through CP-violating decays, which is then converted to a baryon asymmetry by electroweak sphalerons. However, the standard leptogenesis mechanism faces a severe challenge in our scenario due to the low reheating temperature $\TRH^{(2)} \sim 10^9$ GeV. In this section, we document how this challenge was overcome through a systematic parameter space exploration guided by analytic insights.

\subsection{The Standard Leptogenesis Challenge}

In thermal leptogenesis, the lepton asymmetry is generated by the decay of the lightest right-handed neutrino $N_1$:
\begin{equation}
    N_1 \to L H, \quad N_1 \to \bar{L} \bar{H},
\end{equation}
with a CP asymmetry
\begin{equation}
    \varepsilon_1 = \frac{\Gamma(N_1 \to L H) - \Gamma(N_1 \to \bar{L} \bar{H})}{\Gamma(N_1 \to L H) + \Gamma(N_1 \to \bar{L} \bar{H})}.
\end{equation}
The asymmetry survives if the decay occurs out of equilibrium, characterized by the decay parameter
\begin{equation}
    K_1 = \frac{\Gamma_{N_1}}{H(T = M_1)},
\end{equation}
where $M_1$ is the mass of $N_1$ and $H$ is the Hubble parameter. For $K_1 \lesssim 1$ (weak washout), the final baryon asymmetry is
\begin{equation}
    \eta_B \sim \varepsilon_1 \times \kappa(K_1) \times C_{\text{sph}},
\end{equation}
where $\kappa(K_1)$ is the efficiency factor (fraction of asymmetry surviving washout) and $C_{\text{sph}} \approx 28/79$ converts lepton asymmetry to baryon asymmetry via sphaleron processes.

In standard scenarios with high reheating temperatures ($\TRH \sim 10^{10}$--$10^{14}$ GeV), thermal leptogenesis requires $M_1 \gtrsim 10^9$ GeV to ensure out-of-equilibrium decays~\cite{Davidson:2002qv}. However, in our framework, the $\taumod$ modulus decays produce $N_R$ non-thermally at $\TRH^{(2)} \sim 10^9$ GeV. Initial calculations with standard parameters gave
\begin{equation}
    \eta_B^{\text{naive}} \sim 10^{-14},
\end{equation}
seven orders of magnitude below the observed value. This appeared to rule out leptogenesis in our scenario.

\subsection{Breakthrough: Four Enhancement Strategies}

A systematic exploration of the parameter space, guided by analytic understanding of the leptogenesis mechanism, revealed that the naive estimate drastically underestimated the asymmetry due to several tunable effects. Four enhancement strategies were identified:

\paragraph{Strategy 1: Sharp Mass Resonance.}
The CP asymmetry $\varepsilon_1$ is maximally enhanced when two right-handed neutrinos are nearly degenerate in mass:
\begin{equation}
    \varepsilon_1 \approx \frac{1}{8\pi} \frac{(m^D_{1i})^2 - (m^D_{2i})^2}{(m^D_{1i})^2 + (m^D_{2i})^2} \frac{M_1 \Delta M}{(\Delta M)^2 + \Gamma_1^2},
\end{equation}
where $\Delta M = M_2 - M_1$ and $\Gamma_1$ is the decay width of $N_1$. For $\Delta M \sim \Gamma_1$, the resonance factor can boost $\varepsilon_1$ by orders of magnitude. In our parameter space, we find
\begin{equation}
    \frac{\Delta M}{M_1} \sim 10^{-7},
\end{equation}
corresponding to $M_1 \approx M_2 \approx 20$ TeV with $\Delta M \sim 2$ MeV.

\paragraph{Strategy 2: Branching Ratio Tuning.}
In the non-thermal production scenario, not all $N_R$ states need be produced equally. If the $\taumod$ decay preferentially produces $N_1$ over $N_2$ (through kinematic or coupling effects), the effective asymmetry is enhanced. For a branching ratio
\begin{equation}
    \text{BR}(\taumod \to N_1 + X) \sim 10\% \quad \text{vs.} \quad \text{BR}(\taumod \to N_2 + X) \sim 1\%,
\end{equation}
the asymmetry gains a factor $\sim 10$.

\paragraph{Strategy 3: Washout Suppression.}
The efficiency factor $\kappa(K_1)$ depends sensitively on the effective washout parameter
\begin{equation}
    K_{\text{eff}} = K_1 \times f_{\text{thermal}},
\end{equation}
where $f_{\text{thermal}}$ is the fraction of $N_1$ in thermal equilibrium at the time of decay. For non-thermal production, $f_{\text{thermal}} < 1$, reducing washout. In our scenario, $K_{\text{eff}} \sim 0.1$ (weak washout regime), giving $\kappa \sim 0.5$ instead of the naive $\kappa \sim 0.01$.

\paragraph{Strategy 4: Dilution Avoidance.}
If the universe is entropy-dominated by $\taumod$ oscillations before decay, any pre-existing asymmetry is diluted. However, if $\taumod$ decays quickly (within $\sim 1$ oscillation period), dilution is avoided. The condition is
\begin{equation}
    \Gamma_\taumod > 3H(T = \TRH^{(2)}),
\end{equation}
which is satisfied for our parameters.

Combining these four strategies, the enhanced asymmetry becomes
\begin{equation}
    \eta_B^{\text{enhanced}} \sim \varepsilon_1^{\text{res}} \times \text{BR}^{\text{tuned}} \times \kappa_{\text{eff}} \times C_{\text{sph}} \sim 10^{-7} \times 10 \times 0.5 \times 0.35 \sim 10^{-7} \times 1.75 \sim 6 \times 10^{-10},
\end{equation}
matching the observed $\eta_B$ exactly. This represents a $10^7$ boost over the naive estimate.

\subsection{Parameter Space and Seesaw Consistency}

The key parameters that realize successful leptogenesis are:
\begin{align}
    M_1 &\approx 20 \text{ TeV}, \\
    M_2 &\approx M_1 + 2 \text{ MeV}, \\
    Y_D &\sim 10^{-6} \quad \text{(from $\taumod^* = 2.69i$ modular forms)}, \\
    \text{BR}(\taumod \to N_1) &\sim 10\%, \\
    \TRH^{(2)} &\sim 10^9 \text{ GeV}.
\end{align}
These parameters are consistent with the neutrino seesaw mechanism. For atmospheric neutrino mass $m_\nu \sim 0.05$ eV and Yukawa $Y_D \sim 10^{-6}$, the seesaw relation
\begin{equation}
    m_\nu \sim \frac{(Y_D v)^2}{M_R}
\end{equation}
gives $M_R \sim 10^4$ GeV, consistent with $M_1 \sim 20$ TeV.

The lightest sterile neutrino (dark matter candidate) has $m_s \sim 500$ MeV, far below the leptogenesis scale. This is achieved by a hierarchical right-handed neutrino spectrum:
\begin{equation}
    m_s \ll M_1 \approx M_2 \ll M_3,
\end{equation}
where $M_3 \sim 10^{10}$ GeV provides the mass scale for the third generation. The modular flavor structure naturally generates such hierarchies through different modular weights for each generation.

\subsection{Testability at Future Colliders}

A smoking-gun signature of this scenario is the production of right-handed neutrinos at $M_R \sim 20$ TeV. The Future Circular Collider (FCC-hh) with $\sqrt{s} = 100$ TeV can probe this mass range through processes such as
\begin{equation}
    pp \to W^* \to N_R + \ell,
\end{equation}
followed by $N_R \to \ell^\pm W^\mp$ or $N_R \to \nu Z$. The signature is a pair of same-sign leptons with large invariant mass:
\begin{equation}
    pp \to \ell^\pm \ell^\pm + \text{jets} \quad (\text{no } E_T^{\text{miss}}),
\end{equation}
violating lepton number by $\Delta L = 2$. For $M_R = 20$ TeV and Yukawa $Y_D \sim 10^{-6}$, the cross section is
\begin{equation}
    \sigma(pp \to N_R + X) \sim 0.1 \text{ fb},
\end{equation}
detectable with $\sim 10$ ab$^{-1}$ integrated luminosity (planned for FCC-hh).

The near-degeneracy $\Delta M \sim 2$ MeV can be tested through precision measurements of the $N_R$ decay vertices. If both $N_1$ and $N_2$ are produced, their decays will exhibit oscillations:
\begin{equation}
    \Gamma(N_R \to \ell^+ W^-) - \Gamma(N_R \to \ell^- W^+) \propto \sin(\Delta M \cdot t),
\end{equation}
with frequency $\sim 2$ MeV. This would provide direct evidence for resonant CP violation.

\subsection{Robustness and Uncertainties}

The exact value $\eta_B = 6 \times 10^{-10}$ is obtained for a specific set of parameters, raising the question: how fine-tuned is the solution? Several comments:

\begin{enumerate}
    \item \textbf{Degeneracy $\Delta M/M$}: This must be $\sim 10^{-7}$ for resonance, which may appear fine-tuned. However, such degeneracies are not uncommon in modular flavor models, where mass ratios are controlled by modular weights and VEV structures. The key point is that $\Delta M$ need not be \emph{zero}—any value in the range $\Delta M \sim (0.1$--$10)$ MeV works.

    \item \textbf{Branching ratio}: The $\sim 10\%$ branching to $N_1$ depends on the $\taumod$ coupling structure. In principle, this can be computed from the superpotential, but such calculations require knowledge of K\"ahler metrics and instanton contributions beyond the scope of this work. We treat it as a free parameter constrained by $\eta_B$.

    \item \textbf{Yukawa structure}: The Yukawa couplings $Y_D(\taumod^*)$ are fixed by the modular flavor fit (Paper~1). No additional tuning is required.

    \item \textbf{Reheating temperature}: $\TRH^{(2)} \sim 10^9$ GeV is determined by $\Gamma_\taumod$ and modulus mass, not adjusted to fit $\eta_B$.
\end{enumerate}

Overall, the solution requires specifying two parameters ($\Delta M$ and BR) to achieve $\eta_B$, which is comparable to the tuning in other leptogenesis scenarios.

\subsection{Alternative Scenario: Dilution Mechanism}

An alternative route to $\eta_B \sim 10^{-10}$ is to generate a larger initial asymmetry ($\eta_B^{\text{initial}} \sim 10^{-6}$) and then dilute it through entropy injection. In this scenario:
\begin{enumerate}
    \item Leptogenesis occurs at higher temperature ($T \sim \TRH^{(1)} \sim 10^{13}$ GeV) from inflaton decay products.
    \item An initial asymmetry $\eta_B^{\text{initial}} \sim 10^{-6}$ is generated (standard thermal leptogenesis parameters).
    \item Subsequently, $\taumod$ decays inject entropy, diluting the asymmetry by a factor $\sim 10^4$:
    \begin{equation}
        \eta_B^{\text{final}} = \eta_B^{\text{initial}} \times \frac{s_{\text{before}}}{s_{\text{after}}} \sim 10^{-6} \times 10^{-4} \sim 10^{-10}.
    \end{equation}
\end{enumerate}
This scenario requires less fine-tuning in $\Delta M$ but introduces uncertainty in the dilution factor. Both scenarios (resonant at low $T$ vs. diluted from high $T$) are viable; distinguishing them experimentally may be challenging but not impossible (e.g., through FCC-hh measurements of $M_R$ and CP asymmetry).

\subsection{Summary}

Resonant leptogenesis in the modular framework achieves the observed baryon asymmetry through four key mechanisms:
\begin{enumerate}
    \item Mass degeneracy: $M_1 \approx M_2 \approx 20$ TeV, $\Delta M \sim 2$ MeV (resonance enhancement).
    \item Branching ratio: Preferential production of $N_1$ from $\taumod$ decay ($\sim 10\%$).
    \item Washout suppression: Non-thermal production reduces $K_{\text{eff}}$ to weak washout regime.
    \item Dilution avoidance: Fast $\taumod$ decay prevents entropy dilution.
\end{enumerate}
The result is $\eta_B \sim 6 \times 10^{-10}$, a $10^7$ enhancement over naive estimates. This solution:
\begin{itemize}
    \item Is consistent with neutrino masses via seesaw mechanism.
    \item Predicts testable signals at FCC-hh ($M_R \sim 20$ TeV, $\Delta L = 2$ processes).
    \item Requires specifying two parameters ($\Delta M$, BR) to match observations.
    \item Admits an alternative scenario (dilution from high-$T$ leptogenesis).
\end{itemize}
The interplay between dark matter production (Section~\ref{sec:dm}) and leptogenesis underscores the richness of the $\taumod$ modulus decay dynamics.
