\section{Modular Framework from Papers 1--2}
\label{sec:modular}

We briefly review the modular framework established in companion papers, focusing on elements relevant to dark energy.

\subsection{Geometric Origin: $\tau = 2.69i$}

The framework begins with a Calabi-Yau threefold compactification with Hodge numbers $(h^{1,1}, h^{2,1}) = (3, 243)$ and modular group $\Gamma(4)$. The complex structure modulus stabilizes at:
\begin{equation}
\tau = 2.69i
\end{equation}

This value is not arbitrary but emerges from self-consistency: it simultaneously explains 19 flavor observables (Paper 1) and 5 cosmology observables (Paper 2) without any free continuous parameters.

\subsection{Modular Symmetry Breaking}

The modular symmetry $\Gamma(4)$ is broken by $\tau$ stabilization, generating a pseudo-Nambu-Goldstone boson (PNGB). The breaking scale is determined by the geometry:
\begin{equation}
\Lambda = 2.2 \text{ meV}
\end{equation}

This remarkably low scale emerges from:
\begin{equation}
\Lambda \sim \frac{\MPlank}{\text{Vol}(\text{CY})} \times e^{-2\pi |\tau|}
\end{equation}
with $|\tau| = 2.69$ providing exponential suppression.

\subsection{PNGB Quintessence}

The PNGB $\zeta$ from modular breaking has decay constant:
\begin{equation}
f \sim 10^{-3} \MPlank
\end{equation}

Its potential includes instanton contributions weighted by modular forms:
\begin{equation}
V(\zeta) = \Lambda^4 \left[ 1 + k \cos\left(\frac{\zeta}{f}\right) \right]
\end{equation}

The coefficient $k = -86$ is computed from Calabi-Yau instanton actions at $\tau = 2.69i$ (Paper 1, Appendix D). The negative sign is crucial: it makes the minimum at $\zeta \neq 0$, allowing slow roll.

\subsection{Mass from KKLT/LVS}

Moduli stabilization in KKLT~\cite{Kachru2003} or LVS~\cite{Balasubramanian2005} frameworks provides a mass:
\begin{equation}
m_\zeta \sim \frac{\Lambda^2}{\MPlank} \sim 2\times10^{-33}\text{ eV}
\label{eq:m_zeta}
\end{equation}

This exceptionally light mass is essential: $m_\zeta \approx H_0 = 1.5\times10^{-33}$ eV today, placing the field in the frozen quintessence regime.

\subsection{Parameter Summary}

All parameters are determined by $\tau = 2.69i$:
\begin{align}
\Lambda &= 2.2 \text{ meV} \quad \text{(modular breaking scale)} \\
f &= 10^{-3} \MPlank \quad \text{(decay constant)} \\
k &= -86 \quad \text{(instanton coefficient)} \\
m_\zeta &= 2\times10^{-33}\text{ eV} \quad \text{(mass from stabilization)}
\end{align}

These are not free parameters but predictions from the geometry at $\tau = 2.69i$. This is the key difference from phenomenological quintessence models.

\subsection{Connection to Flavor and Cosmology}

The same $\tau = 2.69i$ that determines dark energy parameters also explains:
\begin{itemize}
\item \textbf{Flavor (Paper 1)}: Yukawa hierarchies through modular weights $Y_{ij} \sim \eta(\tau)^{k_i + k_j}$
\item \textbf{Inflation (Paper 2)}: $n_s, r$ through K\"ahler modulus dynamics  
\item \textbf{Dark Matter (Paper 2)}: $\Omega_{DM} h^2$ through reheating temperature
\item \textbf{Dark Energy (this paper)}: $\OmegaDE$ through PNGB quintessence
\end{itemize}

This unified origin from a single modulus value $\tau = 2.69i$ is the central prediction of the framework.
