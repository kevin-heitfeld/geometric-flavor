\section{Framework and Assumptions}
\label{sec:framework}

\subsection{Type IIB Compactification Setup}

We consider Type IIB string theory compactified on the toroidal orbifold $T^6/(\ZZ_3 \times \ZZ_4)$. This geometry provides:
\begin{itemize}
    \item Three complex structure moduli $U^i$ and one axio-dilaton $S = C_0 + i/g_s$\footnote{\textbf{Notation clarification}: Throughout this paper, the symbol $\tau$ refers to the \emph{complex structure modulus} $U_{\text{eff}}$ (the effective value controlling Yukawa couplings), NOT the axio-dilaton $S$. In Type IIB string theory, these are independent moduli. The phenomenologically determined $\tau = 2.69i$ constrains $U_{\text{eff}}$, while the string coupling $g_s = e^\phi$ is determined independently from dilaton stabilization ($g_s \approx 0.10$, see companion paper~\cite{Heitfeld:2025origin}). Some F-theory literature uses $\tau_F$ for the axio-dilaton; we use $S$ to avoid confusion.}
    \item Three K\"ahler moduli $\rho^i$ controlling the CY volume $V = \mathcal{O}(\text{Re}(\rho)^{3/2})$
    \item Sufficient fixed points to accommodate three fermion generations
    \item Discrete Wilson lines enabling hierarchical Yukawa structures
\end{itemize}

The $\ZZ_3 \times \ZZ_4$ orbifold action on $T^6 = T^2 \times T^2 \times T^2$ is defined by simultaneous rotations:
\begin{align}
    \ZZ_3: \quad (z_1, z_2, z_3) &\to (e^{2\pi i/3} z_1, e^{2\pi i/3} z_2, e^{-4\pi i/3} z_3), \\
    \ZZ_4: \quad (z_1, z_2, z_3) &\to (i z_1, i z_2, z_3),
\end{align}
where $z_i$ are complex coordinates on the $i$-th $T^2$. This preserves $\mathcal{N}=1$ supersymmetry in four dimensions and gives Euler characteristic $\chi = -144$ after blow-up resolution of fixed point singularities.

\subsection{D7-Brane Configuration}

We introduce a stack of D7-branes wrapping a four-cycle divisor $\Sigma \subset \CY$ defined by:
\begin{equation}
    \Sigma = w_1 D_1 + w_2 D_2,
\end{equation}
where $D_i$ are basis divisors dual to K\"ahler forms $J_i$, and $(w_1, w_2)$ are integer wrapping numbers. For our specific construction, we choose:
\begin{equation}
    (w_1, w_2) = (1, 1).
    \label{eq:wrapping_choice}
\end{equation}

This choice is \emph{discrete} (not continuously tunable) and determines the topology of the brane embedding. The effective gauge group on the D7-brane worldvolume is SU(5), broken to the SM gauge group by magnetic flux $F$.

\subsection{Topological Invariants}

The wrapping numbers determine several key topological quantities:

\paragraph{Second Chern class.}
For a line bundle $L \to \Sigma$ with first Chern class $c_1(L) = (w_1 J_1 + w_2 J_2)|_\Sigma$, the second Chern class is:
\begin{equation}
    \cctwo = \int_\Sigma c_1(L)^2 = w_1^2 + w_2^2 = 2.
    \label{eq:c2_value}
\end{equation}

\paragraph{Intersection numbers.}
The triple intersection numbers governing Yukawa couplings are:
\begin{equation}
    I_{ijk} = \int_{\CY} J_i \wedge J_j \wedge J_k.
\end{equation}
For $T^6/(\ZZ_3 \times \ZZ_4)$, the non-vanishing intersections are:
\begin{equation}
    I_{333} = \frac{2}{3}, \quad I_{113} = I_{223} = \frac{2}{3}.
\end{equation}
Crucially, the effective intersection number for our wrapped divisor depends on $(w_1, w_2)$:
\begin{equation}
    I_{\text{eff}} = w_1^2 I_{113} + 2 w_1 w_2 I_{123} + w_2^2 I_{223} = \frac{4}{3}.
    \label{eq:intersection_effective}
\end{equation}

\paragraph{Operator basis consistency.}
A key technical point (detailed in Appendix~\ref{app:operator_basis}): $I_{\text{eff}}$ and $\cctwo$ are \emph{not independent} variables. Both are functions of the same wrapping numbers $(w_1, w_2)$, related by:
\begin{equation}
    \frac{\partial I_{\text{eff}}}{\partial w_i} \neq 0 \quad \text{for some } i.
\end{equation}
This means any term like $\cctwo \wedge F$ in an alternative operator basis is not an independent correction but a redefinition already absorbed into intersection numbers. We prove this rigorously in Appendix~\ref{app:operator_basis} via explicit dimensional reduction.

\subsection{Moduli Stabilization}

We assume KKLT-type moduli stabilization \cite{Kachru:2003aw}:

\paragraph{Complex structure and dilaton.}
These are stabilized by flux quantization conditions minimizing the Gukov--Vafa--Witten superpotential:
\begin{equation}
    W_{\text{flux}} = \int_{\CY} G_3 \wedge \Omega,
\end{equation}
where $G_3 = F_3 - \tau H_3$ is the complexified three-form flux and $\Omega$ is the holomorphic three-form. This fixes $U^i$ and $\tau$ with vacuum expectation value $W_0 = \mathcal{O}(1)$--$\mathcal{O}(10)$ (no fine-tuning required).

\paragraph{K\"ahler modulus.}
After complex structure stabilization, the K\"ahler modulus $\rho$ remains flat at tree level. Non-perturbative effects (gaugino condensation on a hidden D7-brane stack or Euclidean D3-instantons) generate:
\begin{equation}
    W_{\text{np}} = A e^{-a\rho},
\end{equation}
where $a = 2\pi/N$ for SU($N$) gaugino condensation. The F-term potential:
\begin{equation}
    V_F = \frac{e^K}{(\text{Im}\,\rho)^2} \left[ |D_\rho W|^2 - 3|W|^2 \right]
\end{equation}
has a supersymmetric AdS minimum at:
\begin{equation}
    \vev{\rho} \sim \frac{1}{a} \ln\left(\frac{A}{W_0}\right).
\end{equation}

\paragraph{De Sitter uplift.}
The AdS vacuum is lifted to de Sitter (small positive cosmological constant) via anti-D3-branes in a warped throat region \cite{Kachru:2003aw}, D-terms \cite{Burgess:2003ic}, or K\"ahler uplift \cite{Balasubramanian:2005zx}. The uplift potential scales as:
\begin{equation}
    V_{\text{up}} \sim \frac{\Delta}{V^\alpha}, \quad \alpha = 2\text{--}3,
\end{equation}
where $\Delta$ is the anti-D3-brane tension or D-term coefficient.

\subsection{Parameter Values and Valid Region}

Our numerical analysis uses:
\begin{align}
    g_s &= 0.10 \quad (\text{string coupling}), \\
    V &= 8.16 \quad (\text{CY volume in string units}), \\
    \tau_2 &= 5.0 \quad (\text{Im}(\tau), \text{ sets instanton suppression}), \\
    W_0 &= 5.0 \quad (\text{flux superpotential}).
\end{align}

These values lie within the KKLT validity region:
\begin{itemize}
    \item $g_s < 0.2$: Perturbative string theory applies
    \item $5 < V < 30$: $\alpha'$ expansion valid, non-perturbative effects relevant
    \item $\tau_2 > 3$: Weakly coupled regime, instantons suppressed
    \item $W_0 \sim \mathcal{O}(1)$--$\mathcal{O}(10)$: Generic flux vacua, no fine-tuning
\end{itemize}

We verify in Appendix~\ref{app:kklt} that varying these parameters within the allowed region changes predictions by $\lesssim 10\%$, demonstrating robustness.

\subsection{Modular Parameter: Physical Vacuum Value}
\label{subsec:tau_vacuum}

Throughout this work, $\tau$ denotes the modular parameter controlling flavor structure. Its physical vacuum value is determined phenomenologically from combined fits to fermion masses and mixing angles:
\begin{equation}
    \tau_* = 2.69\,i.
    \label{eq:tau_vacuum}
\end{equation}
This pure imaginary value lies on a symmetry-enhanced locus in moduli space (the imaginary axis) and is used for all quantitative predictions in this work. The value emerges from the balance condition between competing modular weights across different fermion sectors ($k = 8, 6, 4$ for charged leptons, up-type quarks, and down-type quarks respectively), consistent with the approximate analytic formula $\text{Im}(\tau) \approx 13/\Delta k$ where $\Delta k$ is the spread in modular weights.

\paragraph{Note on parametric control.}
Statements such as ``$\tau_2 \gtrsim 5$'' (Eq.~above) refer to \emph{parametric control} of instanton corrections in the KKLT stabilization mechanism, not the precise vacuum value of the flavor modulus. The distinction is important: $\tau_2 = \text{Im}(\tau_{\text{dilaton}})$ controls string coupling and instantons, while $\tau_* = 2.69i$ is the modular parameter entering Yukawa couplings through modular forms.

\subsection{Origin of Modulus Mass and Stabilization}
\label{subsec:modulus_mass}

A critical question for cosmological viability: \emph{What generates the $\tau$-modulus mass?} Without a mass term, the modulus would be a free field, rolling indefinitely and destroying nucleosynthesis. We outline the stabilization mechanism:

\paragraph{Mass generation.}
The $\tau$-modulus mass arises from three sources in the 4D effective potential:
\begin{enumerate}
    \item \textbf{Flux-induced F-terms:} Complex structure moduli (including $\tau$) are stabilized by the Gukov--Vafa--Witten superpotential $W_{\text{flux}} = \int G_3 \wedge \Omega$. The F-term scalar potential $V_F = e^K |D_\tau W|^2$ generates a mass term after supersymmetry breaking.
    
    \item \textbf{K\"ahler corrections:} At large complex structure ($\text{Im}(\tau) \gg 1$), the K\"ahler potential receives corrections $K \sim -3\ln[\text{Im}(\tau) + \ldots]$. Combined with $W_{\text{flux}}$, this lifts flat directions.
    
    \item \textbf{Nonperturbative effects:} Euclidean D3-instantons (wrapping the same 4-cycle as the flavor D7-brane) contribute $\Delta W \sim A e^{-2\pi \tau}$ to the superpotential. For $\text{Im}(\tau) \sim 3$, these are suppressed by $e^{-6\pi} \sim 10^{-8}$ but nonzero, providing additional stabilization.
\end{enumerate}

\paragraph{Typical mass scale.}
Within KKLT, the $\tau$-modulus mass is parametrically:
\begin{equation}
    m_\tau \sim \frac{m_{3/2}}{\sqrt{\ln(M_{\text{Pl}}/m_{3/2})}},
\end{equation}
where $m_{3/2}$ is the gravitino mass. For $m_{3/2} \sim 10^{13}$ GeV (high-scale SUSY breaking), this gives $m_\tau \sim 10^{12}$ GeV, safely decoupled from all cosmological epochs post-inflation. The modulus settles to its vacuum $\tau_*$ during reheating and remains frozen thereafter.

\paragraph{Clarification: modulus versus modular parameter.}
We emphasize: ``$\tau$'' appears in two distinct contexts:
\begin{itemize}
    \item \textbf{Dynamical modulus field}: The complex scalar $\tau(x)$ in 4D effective theory, with mass $m_\tau \sim 10^{12}$ GeV and VEV $\langle \tau \rangle = \tau_*$.
    \item \textbf{Modular parameter}: The \emph{value} $\tau_* = 2.69i$ that enters modular forms $Y(k_i,\tau)$ in Yukawa matrices. This is the ``frozen'' value, not a free field.
\end{itemize}
In this work, when we write $\tau$ in formulas like $Y_d(\tau)$, we always mean the fixed value $\tau_*$, not a time-dependent field.

\subsection{Explicit Failure Mode: Why $(w_1, w_2) = (2,0)$ Does Not Work}
\label{subsec:failure_mode}

To demonstrate that our construction is \emph{constrained} (not all choices work), we exhibit a concrete failure mode.

\paragraph{Alternative wrapping: $(2,0)$.}
Consider wrapping the D7-brane with numbers $(w_1, w_2) = (2,0)$ instead of $(1,1)$. This is topologically distinct: all flux is concentrated on the first 2-cycle.

\paragraph{Second Chern class.}
The second Chern class is $c_2 = w_1^2 + w_2^2 = 4$ (versus $c_2 = 2$ for $(1,1)$). This appears in the Yukawa suppression factor:
\begin{equation}
    Y_{ij}^{(d)} \sim e^{-\pi c_2 \, \text{Im}(\tau)} \times (\text{modular forms}).
\end{equation}
Larger $c_2$ means stronger exponential suppression.

\paragraph{Why it fails.}
With $c_2 = 4$ and $\text{Im}(\tau) \sim 3$:
\begin{itemize}
    \item Down-type Yukawa eigenvalues are suppressed by $e^{-12\pi} \sim 10^{-16}$.
    \item This predicts $m_b/m_t \sim 10^{-4}$ (versus observed $m_b/m_t \sim 0.02$).
    \item The bottom quark would be $\sim 100\times$ too light: $m_b \sim 30$ MeV instead of 3 GeV.
\end{itemize}
No choice of modular weights or $\tau$ can compensate for this exponential over-suppression. The $(2,0)$ wrapping is \textbf{ruled out} by quark mass data at $>10\sigma$.

\paragraph{Systematic scan.}
We performed a full scan over wrappings $(w_1, w_2)$ with $w_1, w_2 \leq 3$ (12 distinct topologies) in Appendix~\ref{app:wrapping}. Only $(1,1)$ and $(1,2)$ yield $\chi^2/\text{dof} < 3$. The $(1,1)$ choice is unique in achieving $\chi^2/\text{dof} \approx 1$ without fine-tuning $\tau$.

\paragraph{Key takeaway.}
This demonstrates that our framework makes \emph{falsifiable topological predictions}. Not all D7-brane embeddings are compatible with observed flavor structure. The fact that $(1,1)$ works while nearby choices fail is nontrivial evidence that the construction is constrained by data, not by construction.

\subsection{Explicit Statement of Assumptions}
\label{subsec:assumptions}

To ensure complete transparency, we state all assumptions explicitly:

\begin{enumerate}
    \item \textbf{String theory framework:} Type IIB string theory at weak coupling with orientifold projection (O7-planes).
    
    \item \textbf{Compactification geometry:} Toroidal orbifold $T^6/(\ZZ_3 \times \ZZ_4)$ with blow-up resolution of singularities.
    
    \item \textbf{Brane content:} D7-branes with wrapping $(w_1, w_2) = (1,1)$ supporting SU(5) gauge theory.
    
    \item \textbf{Moduli stabilization:} KKLT mechanism with parameters in validity regime stated above.
    
    \item \textbf{Dimensional reduction:} Standard Chern--Simons effective action up to eight-derivative order, following \cite{Jockers:2005zy,Grimm:2005fa}.
    
    \item \textbf{Symmetry breaking:} Magnetic flux $F$ on D7-branes breaks SU(5) $\to$ $SU(3)_c \times SU(2)_L \times U(1)_Y$.
    
    \item \textbf{Matter localization:} Yukawa couplings computed at brane intersection points in unwarped approximation (warping corrections $\lesssim 2\%$, see Appendix~\ref{app:kklt}).
\end{enumerate}

\textbf{What is not assumed:}
\begin{itemize}
    \item We do \emph{not} assume any continuous free parameters in the Yukawa sector.
    \item We do \emph{not} fine-tune moduli to achieve agreement (our values are generic within KKLT regime).
    \item We do \emph{not} select observables post hoc (all 19 SM flavor parameters included).
\end{itemize}

\subsection{Input/Output Classification}

Table~\ref{tab:input_output} clarifies what is input (discrete choices) versus output (derived predictions):

\begin{table}[h!]
\centering
\caption{Classification of inputs and outputs in our framework.}
\label{tab:input_output}
\begin{tabular}{@{}lllll@{}}
\toprule
\textbf{Quantity} & \textbf{Type} & \textbf{Value} & \textbf{Status} & \textbf{Tunable?} \\ \midrule
Orbifold group & Topological input & $\ZZ_3 \times \ZZ_4$ & Discrete choice & No \\
Wrapping $(w_1, w_2)$ & Brane configuration & $(1, 1)$ & Discrete choice & No \\
\midrule
$\cctwo$ & Derived topology & $2$ & Output of inputs & No \\
$I_{ijk}$ & Derived geometry & Table~\ref{tab:intersections} & Output of inputs & No \\
\midrule
$g_s, V, \tau_2, W_0$ & Moduli VEVs & Generic & KKLT stabilization & Within bounds \\
\midrule
19 SM parameters & Physical prediction & Table~\ref{tab:results} & Model output & \textbf{No} \\
$\vev{m_{\beta\beta}}$ & Physical prediction & $10.5 \pm 1.5$ meV & Model output & \textbf{No} \\
\bottomrule
\end{tabular}
\end{table}

This makes clear that while $\cctwo = 2$ follows from discrete choices (not a free parameter), it is indeed an \emph{input choice} rather than a dynamical prediction. Different wrappings would yield different $\cctwo$ values, as we demonstrate in Appendix~\ref{app:wrapping}. The key result is that \emph{given} these discrete choices, all continuous parameters are eliminated.
